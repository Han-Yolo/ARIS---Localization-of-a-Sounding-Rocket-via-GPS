\chapter{GPS Concept for a Sounding Rocket}

The challenge of reaching a GPS accuracy of 1 meter RMS on a sounding rocket consists mainly of two parts.
First, the GPS receiver on the rocket has to be able to get a position fix.
This is not a given because of the extreme conditions the receiver experiences during the flight.
Section \ref{sec:special_conditions} investigates this challenge.
Secondly, the GPS accuracy has to be improved.
The important part here is the vertical accuracy, because this is the value used by the control to reach the right apogee.
Standard GPS is only guaranteed to be accurate to 15 meters for 95 \% of the time in the vertical axis \cite{SPS_Performance}.
A significant improvement is needed to satisfy the accuracy requirement of 1 meter RMS stated in section \ref{sec:requirements}.

\section{Special Conditions on a Sounding Rocket}\label{sec:special_conditions}

The term sounding rockets describes sub-orbital research rockets.
Their time in space is typically only 5-20 minutes.
They fly a parabolic trajectory to bring scientific experiments to altitudes that cannot be studied otherwise.
Regions like the lower ionosphere are to high for weather balloons and to low for satellites.
Sounding rockets fill that gap.
\cite{Sounding_Rockets}

The conditions on a sounding rocket are unlike anything most GPS receivers will ever experience.
The acceleration, velocity and height are the three main parameter that could exceed the operating range of a commercial off-the-shelf (COTS) receiver.
There could either be a technical or legal limit to those parameters.
A receiver as to be picked or tested for the given requirements.

Apart from that, there is also vibration and rotation present on the rocket.
Vibration should not be a problem for a GPS receiver apart from the mechanical stress on the electronics.
Sounding rockets often have a spin during accent to stabilize the flight path.
This rotation can have an effect on how the receiver antenna receives a signal.
There is a problem called the wind-up effect, where an error occurs when the receiver antenna is rotated relative to the satellite antenna.
This error only affects carrier phase measurements and not code measurements.
One full rotation results in an error in the phase measurement of one wavelength, which is about 20 cm for the GPS L1 carrier \cite{Wind_up}. 
So only system with carrier phase measurements would need to correct for this error if the rotation results in a too large error.
Apart from that, the receiver could experience fading of the GPS signals because of rotation.
This could happen if a satellite elevation relative to the antenna changes during the rotation and the antenna gain is not constant in this area.
It will have to be tested how a receiver behaves in such a situation.

Another problem could come from internal interference.
Especially from the telemetry transmitter with its relatively high power.
The GPS signal, when it reaches the surface of the earth, has a signal strength of about -130 dBm.
To pick up such a weak signal, GPS antennas need a high gain and the receivers a high sensitivity.
This increases the risk of interference from a transmitter close by even if it transmits on a different frequency band.
Interference can cause the carrier-to-noise density to shrink to the point that tracking of the GPS signals is lost.
Tests are needed to evaluate the influence of other electronic components in the rocket on the GPS antenna and receiver.
The influence of relatively high power sources like the telemetry transmitter should first be calculated as far as possible to minimize the risk of damaging the GPS receiver.


\section{Accuracy Enhancement Concept}

The process of evaluating the appropriate accuracy enhancement method is explained in this section.
The concept of applying this method to a sounding rocket is then described.
Possible problems in this approach are also examined.

\subsection{Error Correction Method Selection}

The three error correction methods evaluated in chapter \ref{ch:error_correction} were compared to find the right solution for the GPS accuracy problem.
Important factors were the accuracy improvement and the suitability for a sounding rocket.

Dual frequency measurements are a relatively straight forward way of reducing the ionospheric error.
Receivers with the capability of measuring L2C in addition to L1 C/A normally apply this method internally.
From a user perspective, it is a matter of buying the right hardware.
A problem is that at the moment, only about 2/3 of GPS satellites send out the L2C signal, although this will improve with the launch of new satellites.
Also, the error mitigation is limited to ionospheric errors.
Dual frequency measurements alone cannot improve the accuracy to the needed level.

Measurement errors can make up a large part of the total URE depending on the environment and the receiver quality.
Carrier-phase measurements can reduce the multipath error and receiver noise by two orders of magnitude.
But these improvements come at the cost of a more delicate system.
It is even more crucial to continuously track the satellite signals than with standard code measurements.
To get a pseudorange from carrier-phase measurements, the integer ambiguities of the carrier phase have to be resolved first.
This process can take multiple minutes to converge.
If tracking is lost, the ambiguities have to be resolved again.
Extreme conditions like the high acceleration during the burn of a sounding rocket can temporarily lead to a loss of tracking.
In this case, the rocket could potentially only get a carrier-phase GPS fix again after it landed.

The method of differential GPS can reduce all of the external errors.
Satellite errors and atmospheric errors are both correlated for two receivers relatively close together.
Good implementation were able to bring down the position error to the centimeter level.
So theoretically it is possible to achieve the 1 meter accuracy required by the rocket control.
For this reasons, the DGPS approach was chosen for this project.

\subsection{DGPS Concept}

The main challenge with DGPS is to get good corrections from a reference station not too far away.
These corrections can come from an SBAS, over the internet from third party stations, or from a self set up reference station.
Corrections from a self set up reference station were preferred for this project because of multiple reasons.
The reference station can be placed by the ground station to have a short distance to the rocket.
The corrections can then be sent to the rocket over an RF-link as shown in figure \ref{fig:dgps_rocket_concept}.
With an self set up reference station, the corrections can be designed for the specific application.
But with a lot of design freedom also comes a substantial development effort.
A tradeoff will have to be made there between in-house design and COTS.

\begin{wrapfigure}{r}{0.4\textwidth}
  \centering
  \includegraphics[width=0.4\textwidth]{images/DGPS_Rocket_Concept.png}
  \captionof{figure}{DGPS concept for a sounding rocket}
  \label{fig:dgps_rocket_concept}
\end{wrapfigure}

A success of the project is defined by how well the requirements from section \ref{sec:requirements} are met.
The first requirement was about the accuracy.
DGPS was chosen as a methods to reach that goal.
The second requirement said that a position fix should be acquired max. 2 seconds after the burnout.
DGPS itself does not pose a problem to meet that requirement.
Even if the communication link was lost during the burn, the already sent corrections stay valid for some time.
Corrections are correlated up to a minute of time difference or longer and can still be used by the receiver.
It depends on the design and configuration of the receiver in the rocket if this requirement can be met.
The final requirement limits the uplink bandwidth to 2 kbit/s.
With an estimated correction message size of 1 kbit and a rate of one message pre second, only half the bandwidth is used.
The rate of correction messages could also be reduced to once every ten seconds or lower without a big decrease in performance if needed.

A problem could arise from the large height difference between the reference station and the rocket when it climbs to the apogee.
DGPS depends on correlated external errors.
But a height difference of 3 km lets the receiver on the rocket experience a smaller tropospheric delay than the reference station.
DPGS can not account for this difference.
If this turns out to be a big problem for the position accuracy, it has to be accounted for.
Either both the reference station and the rocket receiver apply a tropospheric model to correct their pseudoranges, or a differential tropospheric model is added to the pseudorange corrections.
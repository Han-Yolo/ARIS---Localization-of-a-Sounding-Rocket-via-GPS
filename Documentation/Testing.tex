\chapter{Testing and Validaton}

The implementation described in the previous chapter now has to be tested.
The testing should show if the implemented system produces the required output.
Testing of single parts of the system like the interface from the application to the GPS receiver or the calculation of satellite positions is not the focus here.
The tests described are on the level of the whole system.
Multiple accuracy tests were planned to test different factors that could impact the system.
Most tests were planned to determine how accurate the DGPS system is compared to standard GPS.
Also the influences of the extreme conditiond on a sounding rocket should be determined with these tests as far as possible.
Unfortunately, there was not enough time to conduct all planned tests.
Only static accuracy tests were conducted.
The most representative is described in section \ref{sec:static_accuracy}.


\section{Testing Setup}

The setup to test the system does not differ much from the system described in the implementation chapter.
The difference is that for static accuracy tests, no wireless communication is needed.
The corrections can be sent directy to a second receiver plugged into the laptop which runs the application.
Two receivers are needed for the DGPS, one as reference station and one as user.
Additionally, a receiver has to measure the position without applying the corrections to have a reference to compare the DGPS to.
This setup can be achieved with only two receivers, when the postition output of the reference receiver is used as the uncorrected GPS measurement.

Needed to evaluate the tests are the position estimations with and without applying the corrections, and the PRCs.
The M8T receiver has the ability to log all its output.
It can later be retrieved with the u-blox software u-center.
The logging of the PRCs can be enabled in the DGPS Message Generator application.

Before the logged data can be evaluated, the messages have to be extracted from the bit straem.
A NMEA parser was written in C++ to extact the GxGGA messages and save them to a .csv file.
They contain GPS fix data like UTC time, latitude, longitude and altitude.
The evaluation is done with Matlab scripts that understands the format of the generated .csv files.

\section{Static Accuracy}\label{sec:static_accuracy}

Static accuracy test means that the user receiver does not move and is at the same position as the reference station.
This test was done at a surveyed location to be able to measure the absolute position error.
The Swiss Federal Office of Topography Swisstopo manages the control points data used for natonal surveying.
Swisstopo makes the data available online in form of a map \cite{Swisstopo}.
The choosen test location is a control point on the Sonnenberg in Kriens (Appendix \ref{appendix:control_point_sonnenberg}).


\begin{figure}[ht]
 \centering
 \includegraphics[width=\textwidth]{images/measurement_kriens_setup.png}
 \caption{Measurement setup on the Sonnenberg in Krines}
 \label{fig:measurement_kriens_setup}
\end{figure}


\section{Pending Tests}

\subsection{Mobile Accuracy}
Two receivers test while the rover is moving.
A route is walked which was predetermined with Google Maps.

How accurate is the system when the rover is moving?

\subsection{Horizontal Distance}
Two receiver test with a horizontal distance of about 3km between rover and reference station.

Does DGPS still improve the accuracy if the rover is 3km away from the reference station?

\subsection{Height Difference}

Two receiver test with the rover on a mountain which is about 3km higher than the reference station in the valley.

How much impact does the tropospheric height effect have on the DGPS accuracy?

\subsection{Telemetry Antenna Interference Calculation}

\subsection{Antenna Rotation}

\subsection{Correction Message Interruption}

\subsection{Rocket Test}

\subsection{Evaluation}

Because of the results of the satatic accuracy tests, it would also not make sense to test the other scenarios.
\chapter{Conclusion}

This final chapter looks back at the project in terms of what has be achieved, how the implemented system can benefit a future ARIS project and what other approaches there are to reach the set goals.

\section{Meeting the objectives}

The assignment for for this project was to evaluate errors that impact the accuracy of GPS positioning and find ways to mitigate them.
The most promising approach should be implementet to show its feasability.
Three groups of errors were found that contain errors in the satellite state, atmospheric modelling and signal reception.
From the three evaluated error correction methods, differential GPS emerged to be the most effective and most suited to implement on a sounding rocket.
A DGPS system was planned to implement on a ARIS sounding rocket and the key component that generates the corrections was developed.
The system was successful at generating correction messages that the user receiver can understand and apply.

Requirements were set in the beginning to specify concrete values by which the success of the project can be deteremined.
The first and most important requirement is about the accuracy of the system.
The vertical RMS error of the position estimation for the rocket should be less than 1 meter.
The current state of the implemented system dose not acieve this.
The static accuracy test showed a vertical RMS error of 8.75 meters.
This was slightly better than the standard GPS vertical error of 11.75 meters RMS, but still far off the goal of 1 meter.

The second requirement stated that the rocket should have a GPS fix max. 2 seconds after burnout.
DGPS does not pose a problem to achieve this.
It depends on the choosen GPS receiver.
According to the datasheet, the receiver could loose the fix during burn but should be able to get a fix again within 2 seconds.
Only a test can definitively determine if this requirement can be fulfilled.
During this project, no ARIS rocket was launched with GPS onboard.

Finally, a requirement was set to limit the upload bandwith to the rocket that the system can use.
A max. bandwith of 2 kbit/s was defined.
With the choosen protocol to send the DGPS corrections, only half the max. bandwidth is used when corrections are sent every second, which is plenty.

\section{Future Development}

\section{Different Approaches}



\section{Reflection}


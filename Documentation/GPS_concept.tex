\chapter{GPS Concept and Error Sources}

Along with earlier navigation systems, this chapter explains the positioning concept of GPS and modern Global Navigation Satellite Systems in general.
The error sources that impact the accuracy of GPS are listed and split into categories that determine in which segment those errors occur.

\section{Predecessors}
Before there was GPS, other radionavigation systems were used.
They were mostly ground based and limited to a certan area or time.

An important example is the U.S. system Loran.
It works with multiple transmission stations with a distance of about 1000km to each other.
They send out pulse signals in all directions.
With the difference in time-of-arrival of those pulses, a receiver can triangulate its position.
This method is one variant of hyperbolic positioning.
Only a 2D fix can be acieved with such systems.
The height, if needed, has so be determined with a different method.
The development of Loran-A was started during World War II.
Loran-C is the latest version and still in use today
It has a rms positioning accuracy of about 250m.

Another hyperpolic positioning system was Omega.
It was the first global radionavigation system and operational from 1970 to 1997.
But rather than measuring the time differene of pulses, the phase difference of sinusoidal signals was measured.
This method resulted in a rms postitoning accuracy of 2-4km.
The lower accuracy can be explained with the much wider area it had to cover.

Apart from those ground based systems, there was a working satellite navigation system that came even before Omega.
It is called Transit and was operational from 1964 to 1996.
A doppler based systen that was launched by the U.S. Navy.
In doppler positioning, a 2D position can be determined from the time the doppler of the satellite signal changes from high to low and the sharpnes of the change.
At the moment the doppler changes, the satellite is the closest to the receiver on its orbit.
The distance from the projected orbit can then be determined by how sharp the doppler changes.
When the receiver is directly on the projected orbit, the doppler changes the fastest.
In contrast to modern GNSS, Transit satellites had polar orbits with a low altitude of 1100km.
Only one satellite was visible at a time with a wait time of up to 100 minutes between them.
This made positioning a relatively long process, but the rms positioning accuracy was much better with 25m.


\section{Global Navigation Satellite Systems}

\section{Error Sources}


\chapter*{Abstract}

The Swiss Space Initiative (ARIS) was founded by students from the ETH and HSLU in 2017.
Its inaugural project is to build a sounding rocket to compete in the Spaceport America Cup 2018 in New Mexico.
More than 100 student teams compete there to launch a 4 kg payload to precisely 10'000 feet (ca. 3 km).
In the context of this project called TELL, a system for the localization of a sounding rocket via GPS should be developed.
GPS is evaluated for the special application on a sounding rocket and a system is developed to increase its accuracy.

The sources of errors that impact GPS accuracy were determined and divided into the three groups: Errors in the satellite states, atmospheric errors and signal reception errors.
Different methods were evaluated to mitigate one or more of those errors.
Differential GPS (DGPS) was found to be the most effective method and a proof of concept was implemented to show its functionality.
The system was designed to work with the infrastructure of project TELL.
The ground station serves as the reference station that calculates the differential corrections.
These corrections are sent in the RTCM 2.3 standard over the telemetry RF-link to the rocket.
There, they are applied by the GPS receiver before estimating the rocket position.
The u-blox M8T GPS receiver was used to get GPS raw data.
A C++ application was developed to generate the differential corrections from the GPS raw data at the ground station.

Possible problems for GPS positioning on a sounding rocket were found to be the 10 g acceleration, rotation-related fading and telemetry antenna interference.
The developed DGPS system was tested if it could increase the GPS accuracy in a static test at a survey control point.
The application is able to generate corrections from the GPS raw data that can be understood and applied by the DGPS receiver.
However, the system was not observed to noticeably improve the GPS accuracy.
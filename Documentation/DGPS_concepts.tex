\chapter{Error Correction Methods}

Since GPS became operational, a range of methodes emerged to improve its positioning accuracy, which were not planned by the inventors of GPS.
These methodes are designed to mitigate the effects of specific errors described in \ref{sec:error_sources}.
Three of the most prominent ones are evaluated in this chapter.
They are rated by how effective they are at improving the accuracy and how well they can be implemented on a sounding rocket.


\section{Dual Frequency Measurements}

A relatively new method of error mitigation for civilian GPS are Dual Frequency Measurements.
This only became possible with the launch of Block IIR-M and IIF satellites.
They transmit a second civilian signal on the L2 frequency called L2C.
The benefit of measuring both the L1 C/A and L2C signal is that the ionospheric delay can be determined.
The ionosphere has a different effect on signal on different frequencys.


\section{Carrier-Phase Measurements}

\section{Differential GPS}

\section{Summary}
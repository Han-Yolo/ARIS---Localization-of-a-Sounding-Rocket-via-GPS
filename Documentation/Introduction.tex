\chapter{Introduction}

Global Navigation Satellite Systems, GNSS for short, are uesd in a wide variety of applications nowadays.
The most prominent and oldest of which is GPS.
It started as a military and maritime navigation aid, but is now used extensively in civil applications.
This paper evaluates the use of GPS for the localization of a sounding rocket.

\section{Framework}

The need for such a project originated form ARIS, the Swiss Space Initiative or Akademische Raumfahrt Initiative Schweiz in german.
ARIS was founded in 2017 by students from the ETH Zürich.
Now, over 40 students from the ETH Zürich and the universities of applied sciences in Luzern (HSLU) and Zürich (ZHAW) work on its inaugural project.
The goal of project TELL is to build a sounding rocket to compete in the Spaceport America Cup in New Mexico.
More than 100 student teams compete there to launch a 4kg payload with a sounding rocket to a target altitude of 10000 feet or about 3km \cite{aris}.
A maximum score of 1000 points can be reached.
The points are distributed as follows:
\begin{itemize}
 \item Delivery of Entry Form: 60 points - 6\%
 \item Technical Report: 200 points - 20\%
 \item Design Implementation: 240 points - 24\%
 \item Flight Performance: 500 points - 50\%
\end{itemize}

Points for flight performance are split between apogee relative to the target apogee (350 points) and successful recovery (150 points) \cite{sac_rules2017}.
The points for apogee precision are calculated with this formula:
$$ Points = 350 - \left( \frac{350}{0.3 \times Apogee_{Target}} \right) \times \abs{Apogee_{Target} - Apogee_{Actual}} $$
$$ \text{with } Apogee_{Target} = 3000m$$ \\

For project TELL, GPS positioning is used to locate and recover the rocket.
In project TELL, the rocket motor is dimensioned to overshoot and airbrakes are used to reduce the apogee to the target.
In the air

1\% less points (3.5 points) $\widehat{=}$ 9m deviation


\section{Task Decsription}

\section{Requirements}

\subsubsection{Scoring (1000 points possible)}




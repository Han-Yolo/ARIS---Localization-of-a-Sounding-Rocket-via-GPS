\chapter{Introduction}

Global Navigation Satellite Systems, GNSS for short, are used in a wide variety of applications nowadays.
The most prominent and oldest system is GPS.
It started as a military and maritime navigation aid, but is now used extensively in civil applications.
This thesis evaluates the use of GPS for the localization of a sounding rocket.

\section{Framework}

The need for such a project originated from ARIS \cite{aris}, the Swiss Space Initiative or Akademische Raumfahrt Initiative Schweiz in german.
ARIS was founded in 2017 by students from the ETH Z\"urich and HSLU.
Now, over 50 students from the ETH Z\"urich and the universities of applied sciences in Luzern (HSLU) and Z\"urich (ZHAW) work on its inaugural project which is called TELL.
The goal of project TELL is to build a sounding rocket to compete in the 2018 Spaceport America Cup in New Mexico.
More than 100 student teams compete there to launch a 4 kg payload with a sounding rocket to a target altitude of 10000 feet or about 3 km.

\noindent
\begin{minipage}{0.5\textwidth}
 \centering
 \includegraphics[width=\textwidth]{images/ARIS.png}
 \captionof{figure}{ARIS logo}
\end{minipage}
\begin{minipage}{0.5\textwidth}
 \centering
 \includegraphics[height=5.5cm]{images/SAC_Logo.png}
 \captionof{figure}{Spaceport America Cup logo, Source: spaceportamericacup.com}
\end{minipage}

\vspace{0.5cm}

A jury distributes points for achievements in the competition itself and the planning, engineering and documentation that went into the project.
A maximum score of 1000 points can be reached.
The points are distributed as follows:
\begin{itemize}
 \item Delivery of Entry Form: 60 points - 6 \%
 \item Technical Report: 200 points - 20 \%
 \item Design Implementation: 240 points - 24 \%
 \item Flight Performance: 500 points - 50 \%
\end{itemize}

Points for flight performance are split between reached apogee relative to the target apogee (350 points) and successful recovery (150 points).
The points for apogee precision are calculated with this formula:
$$ Points = 350 - \left( \frac{350}{0.3 \times Apogee_{Target}} \right) \times \abs{Apogee_{Target} - Apogee_{Actual}} $$
$$ \text{with } Apogee_{Target} = 3048 m (10000 feet)$$ 

With that equation, a deviation of about 9 meters from the target apogee results in a loss of 1 \% (3.5 points) of possible points for a correct apogee. \cite{sac_rules2017}

For project TELL, the rocket motor is dimensioned to overshoot the target apogee and air brakes are used to reduce the apogee to the target.
A control loop is implemented to control the position of the air brakes during the accent.
The position and velocity data for the loop is fused from on-board pressure and acceleration sensors.
Although GPS is implemented in the rocket, it is only used to locate and recover the rocket after it landed.
This is because standalone GPS has a vertical error of less than 15 meters for 95 \% of the time. \cite{SPS_Performance}
This is not accurate enough to be beneficial for the sensor fusion.
A vertical error of less than 1 meter root-mean-square (RMS) would be needed for it to be feasible.
With such an accuracy improvement, the GPS position could be fused together with the pressure sensor and accelerometer data.
The addition of GPS would especially help in reducing the drift introduced by the accelerometer.
This is the origin of this bachelor thesis.


\section{Task Decsription}

The task is to evaluate the feasibility of GPS for the localization of a sounding rocket.
This should be done in multiple steps.
First, it has to be determined which internal and external error sources could limit the GPS accuracy.
This means errors that degrade the GPS signal on the way to the receiver, as well as errors that come from the reception of the signal.
A focus should be on potential error sources that are specific to the application in a sounding rocket.

Based on that, methods to mitigate those errors should be found.
These methods should be feasible to implement on a sounding rocket and enhance the accuracy of the GPS positioning to a level where it is usable for the rocket control.
The feasibility of one of those methods should be shown with a proof of concept.
That could be a simulation or a prototype depending on the approach.


\section{Requirements}\label{sec:requirements}

There are two different classes of requirements.
One is to define the performance the system should achieve.
The other is to make it possible to integrate the system into the rocket in terms of interfaces and infrastructure.

The main parameter to rate the performance of a positioning system is the accuracy.
The accuracy consists of two components, the variance and the bias.
Two metrics that are used often are the 95th percentile and the RMS error.
The accuracy requirement for the sounding rocket GPS is:
$$ Error_{RMS} \le 1 m $$

Another metric is the reliability of the system.
This could be a problem especially during the burn of the rocket because of the high acceleration.
For the GPS system to be useful for the competition, it is not vital to have a GPS fix during the burn.
But the rocket should have a fix soon after the burnout to have enough time to correct the flight path before the apogee is reached.
The requirement is:
$$ \textit{GPS fix max. 2 seconds after burnout} $$

In project Tell, a telemetry link is implemented.
This link is mainly a downlink from the rocket to the ground station, but there is the option to add an uplink without much effort.
Such an uplink could potentially be used by the GPS system.
The limitation of the uplink bandwidth for a GPS system is:
$$ \textit{Uplink bandwidth max. 2 kbit/s} $$